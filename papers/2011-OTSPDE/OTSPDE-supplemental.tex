%%%%%%%%%%%%%%%%%%%%%%%%%%%%%%%%%%%%%%%%%%%%%%%%%%%%%%%%%%%%
% First completed draft: 2011/09/01, Serendipity Research
%%%%%%%%%%%%%%%%%%%%%%%%%%%%%%%%%%%%%%%%%%%%%%%%%%%%%%%%%%%%

\documentclass[fleqn,12pt,twoside]{article}
\usepackage[headings]{espcrc1}
\usepackage{amsmath}
\usepackage{graphicx}
\PassOptionsToPackage{ctagsplt,righttag}{amsmath}
\usepackage{multirow}


%******* MACROS *******

% MATH MACROS
\newcommand{\beq}{\begin{equation}}
\newcommand{\eeq}{\end{equation}}
\newcommand{\bea}{\begin{eqnarray}}
\newcommand{\eea}{\end{eqnarray}}

\def\pt{\partial t}
\def\px{\partial x}
\def\py{\partial y}

% NUMERICS MACROS
\def\tu{\tilde{u}}
\def\he{\hat{e}}
\def\dt{\Delta t}
\def\dx{\Delta x}
\def\dy{\Delta y}
\def\dto{\dt_{opt}}

% MISC MACROS
\def\eg{\emph{e.g. }}
\def\ie{\emph{i.e. }}
\def\etal{\emph{et al.}}


%******************************SIZE DECLARATIONS********************************
\topmargin -0.7in


%*******************************DOCUMENT BEGIN**********************************

\begin{document}
\bibliographystyle{unsrt}

\title{Boosting the Accuracy of Finite Difference Schemes via Optimal Time
Step Selection and Non-Iterative Defect Correction \\
(Supplemental Material)}

\author{Kevin T. Chu\address{Serendipity Research, Mountain View, CA 94041, United States}
}

\maketitle

\noindent \rule{6.3in}{1pt}

\appendix
\section{Formal vs. Practical Accuracy
         \label{app:formal_vs_practical_accuracy} }
For time dependent PDEs, evaluating the order of accuracy for a numerical
method is subtle because there are formally two separate orders of accuracy 
to consider:  temporal and spatial.  While it is theoretically
interesting and important to understand how the error depends on both the 
grid spacing $\dx$ and the time step $\dt$, in practice, the accuracy 
of a numerical scheme is always controlled by only one of the two numerical
parameters.  

Even though spatial and temporal orders of accuracy for a numerical method
are formally separate, one will always dominate for a given choice of 
$\dx$ and $\dt$.  For example, when solving the diffusion equation 
using the backward Euler method for time integration with a second-order 
central difference stencil for the Laplacian, formal analysis shows that 
the global error is $O(\dx^2) + O(\dt)$.  Since there are no 
stability constraints on the numerical parameters, the time
step and grid spacing are free to vary independently.  In this situation, the 
practical error for the method depends on the relative sizes of the time step 
and grid spacing.  When $\dt \gg \dx^2$, the practical error is 
$O(\dt)$ which means that the error in the numerical solution is 
primarily controlled by the time step.  Similarly, when 
$\dt \ll \dx^2$, the practical error is $O(\dx^2)$ so that 
the error is controlled by the grid spacing.  Finally, when 
$\dt  = O(\dx^2)$, the practical error is 
$O(\dt) = O(\dx^2)$.  In all cases, the practical error is 
primarily controlled by one of the two numerical parameters, and varying
the subdominant parameter while holding the dominant parameter fixed does 
not significantly affect the error.
 
When there are constraints on the numerical parameters, there is less freedom 
in choosing the controlling parameter.  For instance, if we solve the 
diffusion equation using a forward Euler time integration scheme with a 
second-order central difference stencil for the Laplacian, stability
considerations require that we choose $\dt = O(\dx^2)$.  
Combining this stability constraint with the formal error for the scheme
shows that the practical error is 
$O(\dx^2) + O(\dt) = O(\dx^2)$.  Therefore, the accuracy of the 
method is completely controlled by the grid spacing; the temporal error can 
never dominate the spatial error.


\section{Importance of High-Accuracy for First Time Step for the KPY Scheme
         \label{appendix:KPY_analysis} }
The need for higher-order accuracy when taking the first time step of the KPY 
scheme for the second-order wave equation can be understood by solving the 
difference equation 
\bea
  \he^{n+1} - 2 \he^{n} + \he^{n-1} = \dt^2 \lambda \he^{n}
  \label{eq:error_eqn_normal_mode},
\eea
for the normal modes of the error, where $\he$ is the coefficient of an 
arbitrary normal mode of the spatial operator for the error 
$e \equiv u - \tu$.  Using standard methods for solving linear difference
equations~\cite{carrier_pearson_book} yields the solution
\bea
  \he^n = \he^1 \frac{\kappa_+^n - \kappa_-^n}{\kappa_+ - \kappa_-}
        + \he^0 \kappa_+ \kappa_-
          \frac{\kappa_+^{n-1} - \kappa_-^{n-1}}{\kappa_+ - \kappa_-},
  \label{eq:error_eqn_normal_mode_soln}
\eea
where $\kappa_\pm$ are the roots of the characteristic equation for
(\ref{eq:error_eqn_normal_mode})~\cite{kreiss2002}
\beq
  \kappa_\pm = 1 + \frac{1}{2} \lambda \dt^2
             \pm \dt \sqrt{\lambda + \frac{\lambda^2 \dt^2}{4}}.
\eeq
Notice that the denominator of both terms in
(\ref{eq:error_eqn_normal_mode_soln}) is $O(\dt)$, which implies that the
error in the initial conditions is degraded by one temporal order of accuracy
by the KPY scheme.  Therefore, KPY without OTS-DC requires the first time 
step must be at least third-order accurate; KPY with OTS-DC requires the 
first time step must be at least fifth-order accurate.


%*********************************BIBLIOGRAPHY**********************************

\begin{thebibliography}{10}

\bibitem{carrier_pearson_book}
G.~F. Carrier and C.~E. Pearson.
\newblock {\em Ordinary Differential Equations}.
\newblock Society for Industrial and Applied Mathematics, Philadelphia, 1991.

\bibitem{kreiss2002}
H.-O. Kreiss, N.~A. Petersson, and J.~Ystr\"om.
\newblock {D}ifference {A}pproximations for the {S}econd {O}rder {W}ave
  {E}quation.
\newblock {\em SIAM. J. Numer. Anal}, 40:1940--1967, 2002.

\end{thebibliography}
\end{document}
