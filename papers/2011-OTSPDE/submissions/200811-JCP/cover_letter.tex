\documentclass[12pt]{letter}

% page settings
\textwidth 6.5in
\textheight 9in
\oddsidemargin -0.05in
\evensidemargin -0.05in
\topmargin -0.85in

\begin{document}

\address{}
\signature{Kevin T. Chu}

\begin{letter}
{
% NORMALLY ADDRESS GOES HERE
}

\opening{Dear J. Computational Physics Editor/Referee,}

My article titled \emph{Boosting the Accuracy of Finite Difference Schemes 
via Optimal Time Step Selection} discusses a novel technique for boosting 
the order of accuracy of finite difference schemes for time dependent 
PDEs by choosing an optimal, \emph{fixed} time step.  As discussed in the 
paper, a careful choice of time step can dramatically increase the order of 
accuracy for simple finite difference schemes and make it possible to achieve 
high-order accuracy using only formally low-order finite difference stencils 
and time integration schemes.  For example, optimal time step selection can be 
used to 
\begin{itemize}
\item obtain infinite-order accuracy for the advection equation using 
      a first-order upwind discretization of the gradient operator and
      forward Euler time integration;
\item achieve fourth-order accuracy for the diffusion equation using only 
      formally second-order finite difference stencils and forward Euler 
      time integration. 
\end{itemize}
This paper describes the method for calculating the optimal time step and 
explains the observed boost in accuracy using straightforward numerical 
analysis arguments.  The technique of optimal time step selection is 
applicable to linear and semilinear PDEs in any number of space dimensions 
and both regular and irregular domains.  Examples of each of these types of 
problems are discussed and analyzed in the paper.

Thank you for your review of the paper.  I look forward to hearing your 
comments and suggestions for improvement of the manuscript.

\closing{Sincerely,}

\end{letter}

\end{document}
