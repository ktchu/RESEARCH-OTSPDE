%%%%%%%%%%%%%%%%%%%%%%%%%%%%%%%%%%%%%%%%%%%%%%%%%%%%%%%%%%%%
% Manuscript started: 2008/11/24
% First completed draft: 2008/??/??
% Submission draft: 2008/??/??
%%%%%%%%%%%%%%%%%%%%%%%%%%%%%%%%%%%%%%%%%%%%%%%%%%%%%%%%%%%%

\documentclass[fleqn,12pt,twoside]{article}
\usepackage[headings]{espcrc1}
\usepackage{amsmath}
\usepackage{graphicx}
\PassOptionsToPackage{ctagsplt,righttag}{amsmath}
\usepackage{multirow}


%********************************** MACROS ************************************

% COMMON EXPRESSIONS
\def\eg{\emph{e.g., }}
\def\ie{\emph{i.e., }}
\def\etal{\emph{et al.}}
\def\etc{\emph{etc.}}

% UTILITY COMMANDS
\newcommand{\bc}{\begin{center}}
\newcommand{\ec}{\end{center}}
\newcommand{\beq}{\begin{equation}}
\newcommand{\eeq}{\end{equation}}
\newcommand{\bea}{\begin{eqnarray}}
\newcommand{\eea}{\end{eqnarray}}
\newcommand{\ba}{\begin{array}}
\newcommand{\ea}{\end{array}}

% NOTATION
\def\div{\ensuremath{\nabla \cdot}}
\def\grad{\ensuremath{\nabla}}
\def\lapl{\ensuremath{\nabla^2}}
\def\bilapl{\ensuremath{\nabla^4}}

% PDE MACROS
\def\pt{\partial t}
\def\px{\partial x}
\def\py{\partial y}
\def\pz{\partial z}
\def\tU{\tilde{U}}
\def\tP{\tilde{P}}
\def\tu{\tilde{u}}
\def\tv{\tilde{v}}
\def\tw{\tilde{w}}

% NUMERICS MACROS
\def\dt{\Delta t}
\def\dx{\Delta x}
\def\dy{\Delta y}
\def\dz{\Delta z}
\def\dto{\dt_{opt}}


%****************************** SIZE DECLARATIONS *****************************
\topmargin -0.7in


%****************************** DOCUMENT BEGIN ********************************

\begin{document}
\bibliographystyle{unsrt}

\title{A Fourth-Order Finite Difference Scheme for the 
Incompressible Navier-Stokes Equations via Optimal Time Step Selection}

\author{
K. T. Chu\address{Vitamin D, Inc., Menlo Park, CA 94025}$^,$\address[IHPC]{Institute of High Performance Computing, A*STAR, Singapore, Singapore}
%D. V. Le\addressmark[IHPC],
%and K. H. Chiam\addressmark[IHPC]
}

\runtitle{Fourth-Order Navier Stokes Scheme via Optimal Time Step Selection}
\runauthor{??}

\maketitle

\noindent \rule{6.3in}{1pt}

\begin{abstract}
place holder...
\end{abstract}

\noindent \rule{6.3in}{1pt}

% KEYWORDS
% optimal time step, Navier-Stokes equations, computational fluid dynamics

\section{Introduction}
place holder...
Prior work~\cite{bruger_2005,shukla_2007}

Optimal time step selection~\cite{chu_2009}


\section{Method}

\subsection{Highlights}
The propsed fourth-order scheme for the incompressible Navier-Stokes equations
only requires forward Euler time integration and a fourth-order accurate
solution of the the pressure correction equation.  Does not require 
correction terms that involve spatial or temporal derivatives higher than
those that are present in the original equations.

There are two main complication in deriving the scheme:
(1) choosing an appropriate discretization for the nonlinear convection
term and 
(2) calculating the correction terms in the forward Euler scheme.  These
two derivations are closely interrelated.

We can use a standard staggered grid configuration with $\dx = \dy = \dz $ 
to avoid unphysical oscillations in the pressure.


\subsection{Method}
Incompressible Navier-Stokes equations:
\bea
  \frac{\partial U}{\pt} + \left(U \cdot \grad \right) U 
    &=& -\grad P + \frac{1}{Re} \lapl U + f
    \\
  \div U &=& 0
\eea
Pressure equation:
\bea
  \lapl P = \div f - \div \left( (U \cdot \grad) U \right)
\eea

Let variables with and without tildes denote the true and numerical solution,
respectively.


\subsubsection{Time discretization}
Forward Euler time integration for $U$:
\beq
  \frac{\partial U^n}{\pt} = \frac{U^{n+1}-U^{n}}{\dt} 
\eeq
True solution satisfies
\beq
  \frac{\partial \tU^n}{\pt} = 
    \frac{\tU^{n+1}-\tU^{n}}{\dt} - \frac{\dt}{2} \tU^n_{tt} + O(\dt^2)
  \label{eq:forward_Euler_error}
\eeq

For the Laplacian term, we use any stencil which has an isotropic error:
\beq
  \lapl \tU = L_{iso} \tU - \frac{\dx^2}{12} \bilapl \tU + O(\dx^4).
\eeq
In 2D, the standard nine-point stencil works fine.  In 3D, any of the
isotropic stencils in Patra and Karttunen~\cite{patra_2005} would work.

As with the diffusion equation, the optimal time step $\dt_{opt}$ will
come from a balance between the bilaplacian terms that appear in the 
leading-order terms in the spatial and temporal errors.  This leads to 
an optimal time step $\dt_{opt} = Re \dx^2/6$.

The discretization of the other spatial derivatives will be specially
derived so that they exactly cancel out terms in the leading-order temporal
error when the optimal time step is used.

From (\ref{eq:forward_Euler_error}), we see that the leading-order temporal
error involves $\tU_{tt}$:
\bea
  \tU_{tt} &=&
    -\left(\tU_t \cdot \grad \right) \tU 
    -\left(\tU \cdot \grad \right) \tU_t 
    -\grad P_t
    + \frac{1}{Re} \lapl \tU_t
    + f_t
    \nonumber \\
    &=& \frac{1}{Re^2} \bilapl \tU - \frac{1}{Re} \lapl \grad P  
    \nonumber \\
    & & - \frac{1}{Re} 
    \left[ \lapl \left( (\tU \cdot \grad) \tU \right)
         + \left(\tU \cdot \grad \right) \lapl \tU 
         + \left(\lapl \tU \cdot \grad \right) \tU 
    \right]
    \nonumber \\
    & & + \left(\tU \cdot \grad \right) \left(\tU \cdot \grad \right) \tU
    + \left( (\tU \cdot \grad) \tU \cdot \grad \right) \tU
    \nonumber \\
    & & - \left(\tU \cdot \grad \right) \left(-\grad P + f \right)
        - \left((-\grad P + f) \cdot \grad \right) \tU
    \nonumber \\
    & & -\grad P_t + f_t + \frac{1}{Re} \lapl f
\eea

This expression looks messy, but it turns out that the first two lines
can be partially cancelled out by careful choice of the spatial 
discretizations.  The last three lines are correction terms that need
to be added to the RHS of the forward Euler scheme.  A few of the terms 
are already computed (\eg $-\grad P + f$) and can be reused to reduce the
computational cost of evaluating the correction terms.  For a heuristic
way to see whether a term is can be cancelled out or must be explicitly 
included as a correction term, it is helpful to look at the analysis of 
the viscous Burgers equation in~\cite{chu_2009}.

To make it easier to derive the spatial discretizations, it is convenient
to expand the cancellation terms in component form.  The $x$-component 
of 
\bea
  \lapl \left( (\tU \cdot \grad) \tU \right)
  + \left(\tU \cdot \grad \right) \lapl \tU 
  + \left(\lapl \tU \cdot \grad \right) \tU 
\eea
is given by 
\bea
  \lapl \left( (\tU \cdot \grad) \tu \right)
  &+& \left(\tU \cdot \grad \right) \lapl \tu 
  \ + \ \left(\lapl \tU \cdot \grad \right) \tu =  
  \nonumber \\
  &2& \left[ 
  \begin{array}{c}
    \tu \lapl \tu_x + \tv \lapl \tu_y + \tw \lapl \tu_z \\
    + \ \grad \tu_x \cdot \grad \tu 
    + \grad \tu_y \cdot \grad \tv
    + \grad \tu_z \cdot \grad \tw \\
    + \ \tu_x \lapl \tu + \tu_y \lapl \tv + \tu_z \lapl \tw 
  \end{array}
  \right]
%  \nonumber \\
%  &2& \left[ 
%  \begin{array}{c}
%    \tu \lapl \tu_x + \tv \lapl \tu_y + \tw \lapl \tu_z \\
%    + \div \left( \tu_x \cdot \grad \tu \right)
%    + \div \left( \tu_y \cdot \grad \tv \right)
%    + \div \left( \tu_z \cdot \grad \tw \right)
%  \end{array}
%  \right]
  \label{eq:u_temporal_error}
\eea
Similarly, the $x$-component of the first pressure term in $\tU_{tt}$ is
\beq
  -\frac{1}{Re} 
  \left( P_{xxx} + P_{xyy} + P_{xzz} \right)
  \label{eq:P_temporal_error}
\eeq
The $y$- and $z$-components have analogous forms.

The leading-order term in the local truncation error for the forward
Euler scheme is
\beq
 e_{spatial} \dt + \frac{\dt^2}{2} \tU_{tt},
\eeq
so we seek finite difference discretizations of the spatial derivatives that
will exactly balance some of the terms in the temporal error.

For the pressure gradient term, use a second-order central-difference 
approximation.  In the $x$-direction:
\bea
  \frac{\partial P_{i+1/2,j,k}}{\px} &=& 
  \alpha \left(\frac{P_{i+1,j,k}-P_{i,j,k}}{\dx}\right)
  \nonumber \\
  &+& \frac{\beta}{2} \left( \frac{P_{i+1,j+1,k}-P_{i,j+1,k}}{\dx} 
                           + \frac{P_{i+1,j-1,k}-P_{i,j-1,k}}{\dx} 
                      \right)
  \nonumber \\
  &+& \frac{\beta}{2} \left( \frac{P_{i+1,j,k+1}-P_{i,j,k+1}}{\dx} 
                           + \frac{P_{i+1,j,k-1}-P_{i,j,k-1}}{\dx} 
                      \right),
\eea
with $\alpha + 2 \beta = 1$.
The discretization error for this scheme is 
\beq
  - \left(\alpha + 2 \beta \right ) \frac{\dx^2}{12} \tP_{xxx}
  - \beta \frac{\dx^2}{4} + \tP_{xyy}
  - \beta \frac{\dx^2}{4} + \tP_{xzz}
  + O(\dx^4),
\eeq
where we have use the fact that the grid spacing is the same in all
directions.  Therefore, to cancel out the contribution to the truncation
error from (\ref{eq:P_temporal_error}) when $\dt = \dt_{opt}$, we want to 
choose $\alpha = 1/3 = \beta$.  For a 2D problem, we would choose 
$\alpha = 2/3$ and $\beta = 1/3$.  

For the nonlinear convection term, we can eliminate the first three terms in
(\ref{eq:u_temporal_error}) by computing gradients using an analogous scheme 
as for the pressure gradient.  Next, we can carefully choose the approximation 
of $\hat{u}_{i+1/2,j,k}$, $\hat{v}_{i+1/2,j,k}$, and $\hat{w}_{i+1/2,j,k}$ 
so that the fourth term is completely eliminated and the error when computing
$v u_y + w u_z$ are automatically eliminated by adding a multiple of the
$u_y \lapl v + u_z \lapl w$ terms in (\ref{eq:u_temporal_error}).
The remaining three terms need to be directly added as a correction term.  
The can be computed using things that have already been computed anyways (I 
think).  For the $x$-component of the fluid velocity, we use the following 
finite difference approximations for the gradient:
\bea
  \left. \frac{\partial \tu}{\px}\right|_{i+1/2,j,k} &=& 
  \alpha \left(\frac{u_{i+3/2,j,k}-u_{i-1/2,j,k}}{2\dx}\right)
  \nonumber \\
  &+& \frac{\beta}{2} \left( \frac{u_{i+3/2,j+1,k}-u_{i-1/2,j+1,k}}{2\dx} 
                           + \frac{u_{i+3/2,j-1,k}-u_{i-1/2,j-1,k}}{2\dx} 
                      \right)
  \nonumber \\
  &+& \frac{\beta}{2} \left( \frac{u_{i+3/2,j,k+1}-u_{i-1/2,j,k+1}}{2\dx} 
                           + \frac{u_{i+3/2,j,k-1}-u_{i-1/2,j,k-1}}{2\dx} 
                      \right).
\eea
\bea
  \left. \frac{\partial \tu}{\py}\right|_{i+1/2,j,k} &=& 
  \alpha \left(\frac{u_{i+1/2,j+1,k}-u_{i+1/2,j-1,k}}{2\dy}\right)
  \nonumber \\
  &+& \frac{\beta}{2} \left( \frac{u_{i+3/2,j+1,k}-u_{i+3/2,j-1,k}}{2\dy} 
                           + \frac{u_{i-1/2,j+1,k}-u_{i-1/2,j-1,k}}{2\dy} 
                      \right)
  \nonumber \\
  &+& \frac{\beta}{2} \left( \frac{u_{i+1/2,j+1,k+1}-u_{i+1/2,j-1,k+1}}{2\dy} 
                           + \frac{u_{i+1/2,j+1,k-1}-u_{i+1/2,j-1,k-1}}{2\dy} 
                      \right).
\eea
\bea
  \left. \frac{\partial \tu}{\pz}\right|_{i+1/2,j,k} &=& 
  \alpha \left(\frac{u_{i+1/2,j,k+1}-u_{i+1/2,j,k-1}}{2\dz}\right)
  \nonumber \\
  &+& \frac{\beta}{2} \left( \frac{u_{i+3/2,j,k+1}-u_{i-1/2,j,k-1}}{2\dz} 
                           + \frac{u_{i+3/2,j,k+1}-u_{i-1/2,j,k-1}}{2\dz} 
                      \right)
  \nonumber \\
  &+& \frac{\beta}{2} \left( \frac{u_{i+1/2,j+1,k+1}-u_{i+1/2,j+1,k-1}}{2\dz} 
                           + \frac{u_{i+1/2,j-1,k-1}-u_{i+1/2,j-1,k-1}}{2\dz} 
                      \right),
\eea
with $\alpha = 1/3 = \beta$.
For a 2D problem, we use analogous formulae with 
$\alpha = 2/3, \beta = 1/3$.

In 3D, we use the following for
$\hat{u}_{i+1/2,j,k}$, $\hat{v}_{i+1/2,j,k}$, and $\hat{w}_{i+1/2,j,k}$, 
\bea
  \hat{u}_{i+1/2,j,k} = \frac{1}{6}
  \left[
      u_{i+3/2,j,k} + u_{i-1/2,j,k} 
    + u_{i+1/2,j+1,k} + u_{i+1/2,j-1,k} 
    + u_{i+1/2,j,k+1} + u_{i+1/2,j,k-1}
  \right]
\eea
\bea
  \hat{v}_{i+1/2,j,k} &=& 
  \frac{3}{16}
  \left[
    \begin{array}{c}
    \ \ v_{i+1,j+1/2,k} + v_{i,j+1/2,k} \\
    + v_{i+1,j-1/2,k} + v_{i,j-1/2,k}
    \end{array}
  \right] \nonumber \\
  &+& \frac{1}{32}
  \left[
    \begin{array}{c}
    \ \ v_{i+1,j+1/2,k+1} + v_{i,j+1/2,k+1} \\
    + v_{i+1,j-1/2,k+1} + v_{i,j-1/2,k+1} \\
    + v_{i+1,j+1/2,k-1} + v_{i,j+1/2,k-1} \\
    + v_{i+1,j-1/2,k-1} + v_{i,j-1/2,k-1}
    \end{array}
  \right]
\eea
\bea
  \hat{w}_{i+1/2,j,k} &=& 
  \frac{3}{16}
  \left[
    \begin{array}{c}
    \ \ w_{i+1,j,k+1/2} + w_{i,j,k-1/2} \\
    + w_{i+1,j,k+1/2} + w_{i,j,k-1/2}
    \end{array}
  \right] \nonumber \\
  &+& \frac{1}{32}
  \left[
    \begin{array}{c}
    \ \ w_{i+1,j+1,k+1/2} + w_{i,j+1,k+1/2} \\
    + w_{i+1,j+1,k-1/2} + w_{i,j+1,k-1/2} \\
    + w_{i+1,j-1,k+1/2} + w_{i,j-1,k+1/2} \\
    + w_{i+1,j-1,k-1/2} + w_{i,j-1,k-1/2} \\
    \end{array}
  \right]
\eea
In 2D, the analogous expressions are:
\bea
  \hat{u}_{i+1/2,j} = \frac{1}{3} u_{i+1/2,j}
  + \frac{1}{6}
  \left[
      u_{i+3/2,j} + u_{i-1/2,j} 
    + u_{i+1/2,j+1} + u_{i+1/2,j-1} 
  \right]
\eea
\bea
  \hat{v}_{i+1/2,j} &=& 
  \frac{1}{4}
  \left[
      v_{i+1,j+1/2} + v_{i,j+1/2} 
    + v_{i+1,j-1/2} + v_{i,j-1/2}
  \right] 
\eea

In the 3D case, the discretization error for $-(U \cdot \grad) u$ is
\bea
  \frac{\dx^2}{6}
  \left[\tu \lapl \tu_x + \tv \lapl \tu_y + \tw \lapl \tu_z \right]
  + \frac{\dx^2}{6} \tu_x \lapl \tu 
  + \frac{\dx^2}{8} \left[ \tu_y \lapl \tv + \tu_z \lapl \tw \right]
  \label{eq:convection_err_3d}
\eea
Therefore, when the optimal time step is used the local truncation error
associated with the convection terms is:
\bea
  -\frac{\dto \dx^2}{24} \left[ \tu_y \lapl \tv + \tu_z \lapl \tw \right]
  -\frac{\dto^2}{Re}
    \left[ \grad \tu_x \cdot \grad \tu
         + \grad \tu_y \cdot \grad \tv
         + \grad \tu_z \cdot \grad \tw
    \right] = 
    \nonumber \\
  -\frac{\dto^2}{Re} 
  \left[ \frac{1}{4} \left( \tu_y \lapl \tv + \tu_z \lapl \tw \right)
  +  \grad \tu_x \cdot \grad \tu
  + \grad \tu_y \cdot \grad \tv
  + \grad \tu_z \cdot \grad \tw
  \right]
\eea
This is the correction term that is required for the convection term's
contribution to the local truncation error.

In the 2D case, the discretization error for $-(U \cdot \grad) u$ is
\bea
  \frac{\dx^2}{6}
  \left[\tu \lapl \tu_x + \tv \lapl \tu_y \right]
  + \frac{\dx^2}{6} \tu_x \lapl \tu 
  + \frac{\dx^2}{8} \tu_y \lapl \tv 
\eea
and an analogous result is obtained for the convection term's contribution
to the correction term.

The discretizations of the $y$- and $z$- components of the Navier-Stokes
equations are analogous.


\subsubsection{Pressure discretization}
Use a compact fourth-order scheme for the pressure Poisson equation (or 
Poisson equation for the pressure correction).  To ensure that the RHS 
is fourth-order accurate, compute the $\div f$ term analytically or (use
a fourth-order discretization ... this will probably require wider stencils).
For the $\div \left( \left(U \cdot \grad \right) U \right)$ term, we can
use the same stencil as for the time advance and explicitly subtract
the discretization error (\ref{eq:convection_err_3d}) to get fourth-order
accuracy.


\section{Concerns/Thoughts} 

\begin{itemize}
\item I have some concerns that while the the above discretization is
fourth-order accurate, if we only have a fourth-order accurate solution for 
the pressure, the accuracy of the gradient will only be third-order accurate
leading to an overall third-order accurate scheme.  It may work out that this
is not the case, but it's something to look out for.  A more careful 
analysis of the pressure scheme may be necessary to determine if this will 
affect the overall scheme.

\item It would be nice to think of an interpretation of the fancy spatial
stencils as cancelling out artificial dissipation or something like that.

\item If it's acceptable to have pressure oscillations (because we only
case about the fluid velocity), then using an unstaggered grid may be nice
because we can use the same simple formula for 
$\hat{u}$, $\hat{v}$, and $\hat{w}$ which leads to a discretization error
for the convection term that has errors that are the same in all directions:
\beq
  \frac{\dx^2}{6}
  \left[\tu \lapl \tu_x + \tv \lapl \tu_y + \tw \lapl \tu_z \right]
  + \frac{\dx^2}{6} 
  \left[ \tu_x \lapl \tu + \tu_y \lapl \tv + \tu_z \lapl \tw \right]
\eeq
so that six of the terms in (\ref{eq:u_temporal_error}) can be eliminated
by using the optimal time step approach.  The convection contribution to 
the correction term in this case is:
\beq
  -\frac{\dto^2}{Re} 
  \left[ \grad \tu_x \cdot \grad \tu
       + \grad \tu_y \cdot \grad \tv
       + \grad \tu_z \cdot \grad \tw
  \right]
\eeq
The formulas for the pressure gradient discretization would also have be 
adjusted a little bit, but it would end up being basically the same as the
gradient expression for the convection term.

\end{itemize}


\section{\label{sec:summary} Summary} 


\section*{Acknowledgments}
The author gratefully acknowledges the support of Vitamin D, Inc.
and the Institute for High-Performance Computing (IHPC) in Singapore. 
The author would like to thank ??  for helpful suggestions on the manuscript.  

\bibliography{OTS-NavierStokes}

\end{document}
