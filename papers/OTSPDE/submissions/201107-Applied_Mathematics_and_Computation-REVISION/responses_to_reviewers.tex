\documentclass[12pt]{article}

% page settings
\textwidth 6.5in
\textheight 9in
\oddsidemargin -0.05in
\evensidemargin -0.05in
\topmargin -0.85in

\begin{document}

\section*{Responses to Reviewers}

\subsection*{List of Changes to Manuscript}

\begin{enumerate}

\item Throughout manuscript (including the title), clarified and emphasized the
      general need for OTS and non-iterative defect correction in order to
      achieve the boost in the order of accuracy.  Changed OTS-DC to OTS-NIDC
      throughout manuscript.

\item Streamlined discussion throughout the manuscript to "shorten the verbiage
      without altering the message". Reorganized figures a little bit to reduce
      length of manuscript.

\item Moved comparisons with other methods out of the introduction section into
      its own section titled "Theoretical Comparisons with Related Numerical
      Techniques"

\item Modified the discussion in Section 2.2 (previously Section 1.2) to
      emphasize that OTS-NIDC is an important special case of the modified
      equation method where regularization of the modified equation is
      automatic via parameter choice rather than via iterative procedures.

\item Removed claims that optimal selection of numerical parameters is not 
      common.  When I originally wrote this manuscript (2007), I did not find
      many examples of optimal numerical parameter selection.  Apparently the
      situation has changed.

\item Removed discussion about applicability to GPU computing except for a
      brief comment that investigation of benefits of OTS-NIDC to GPU computing
      is an area for future research.

\item Throughout the manuscript, added clarification that theoretical costs are
      asymptotic.

\item Added performance data for 1D Burgers equation and a brief discussion of
      its interpretation.

\item Added performance data for 2D diffusion equation on an irregular domain.

\item Added emphasis on the accuracy limitations of the Crank-Nicholson schemes
      for the fourth-order parabolic equation.

\item Renamed 'Summary' section to 'Conclusions' and revised second half of
      section.

\item Minor grammatical revisions throughout manuscript.

\end{enumerate}

\subsection*{Reviewer \#2}

\subsubsection*{General rebuttal}

On the point that OTS-NIDC is "technically a special case of the method of
Modified Equation," I offer the perspective that it is an important special
case because the singular perturbations in the modified equations are
regularized \emph{automatically} by the choice of optimal time step.  The
resulting modified equation is a \emph{regular} perturbation to the original
PDE, so there is no need for iterative refinement to handle the singular
perturbation.

\subsubsection*{Rebuttals to specific points}

\begin{enumerate}
\item Throughout the manuscript, acronyms have been defined before use.

\item Throughout the manuscript, the identified typos have been fixed.  The
      appendices have been removed to reduce the length of the manuscript.

\item The comparison with other numerical methods has been moved to its own 
      section titled "Theoretical Comparisons with Related Numerical
      Techniques".  All other sections and references have been renumbered.

\item The suggested references have been added.

\item The last section has been renamed "Conclusions" as suggested.
      Discussion of future directions for research has been added to the 
      conclusions section.
   
\item The entire manuscript has been rechecked again for missed mistakes and
      reorganized to improve clarity and reduce length of manuscript.

\end{enumerate}


\subsection*{Reviewer \#3}

\subsubsection*{General rebuttal}

I agree that the OTS-NIDC concept contains elements that are very old ideas. 
However, I have personally only seen the method used as a 'trick' in special
cases.  I have not seen a unified presentation that discusses its utility as
a general approach to boosting the order of accuracy of finite difference
schemes without requiring derivation and use of formally high-order
discretizations, etc.  This perspective could certainly be a result not having
seen \emph{all} of the extensive literature on numerical methods.  That said,
the approach I present does not appear to be employed as often as I believe it
could be.  Perhaps the ideas, while old, could use a refresher in the
collective consciousness of computational mathematicians and scientists.

I also agree that there is additional analytical and coding work
that needs to be done in order to use OTS-NIDC.  However, the extra work and
coding complication is not as onerous as it may seem at first glance.  For
example, in the process of implementing the original numerical scheme, it is
likely that several of the required differential operators have already been 
implemented (hopefully even coded as functions to make them easy to reuse).  
Also, almost any time one wants to improve the accuracy or computational
performance of a numerical scheme (e.g.  modified equation method, formally
higher-order discretizations, implicit time integration), additional work is
likely to be required.

On the issue of cost overhead, there is certainly a cost overhead with adding 
correction terms.  However, the boost in the order of accuracy results in an
overall savings in computational effort for a given desired level of accuracy.
Throughout the manuscript, I have attempted to emphasize this point by plotting
computational performance as a function of numerical error (instead of as a
function of the number of grid points).

On the comment that the results shown in Figure 4c (previously Figure 6)
indicate that the cost overhead of the defect correction terms make the method 
quite demanding, I have to disagree.  The actual reason for the poor
performance of OTS-NIDC scheme relative to the Crank-Nicholson schemes for the
fourth-order parabolic equation is \emph{not} the cost of the correction term. 
While the computational cost correction terms certainly widens the gap, 
the correction terms themselves ($f_xxxx$ and $f_t$) are not that complicated or
costly to compute.  Rather, it is the extreme smallness of the prefactor of
the optimal time step which is 7/120 combined with the low cost of inverting
the bilaplacian in 1D.  The small prefactor on the time step also afflicts the
non-OTS-NIDC forward Euler scheme which is constrained to have a time step 
smaller than $3 (\Delta x)^4 / 40 \kappa$.  This point is discussed in second
to last paragraph of Section 4.5 (previously Section 3.5).

When analyzing the performance of the various schemes for the fourth-order
parabolic equation, it is worth noting that it is important to take the
accuracy results into consideration.  As Figure 4b and 4c (previously Figures
5 and 6) illustrate, the Crank-Nicholson schemes suffer from a limitation in
the achievable accuracy as a result of round-off errors introduced by matrix
inversion.  The forward Euler schemes (including OTS-NIDC) do not suffer from
this problem.

On the point of applicability to systems of nonlinear PDEs, I am currently 
working on applying the method to systems of PDEs and the Navier-Stokes
equations.  I have completed work on the former and can confidently report that
with one additional simple idea, OTS-NIDC works for systems of PDEs.  My work
on the latter looks promising, but I have not fully completed the analysis or
the coding required to demonstrated the theoretical performance estimates.

\subsubsection*{Rebuttals to specific points}

\begin{enumerate}

\item I have added measured performance results for the 1D viscous Burgers
      equation and the 2D diffusion equation on an irregular domain.  The 
      measured performance results now include: 1D viscous Burgers equation,
      1D fourth-order parabolic equation, 2D diffusion equation on a regular 
      domain, and 2D diffusion equation on an irregular domain.  I did not 
      include performance results for the 1D second-order wave equation because
      that simulation ran so fast that the timing results contained too much 
      noise.

      I agree there are many alternative methods besides OTS-NIDC for boosting
      the order of accuracy of linear problems.  My manuscript does not claim
      to be better than those other methods.  It just presents an alternative
      method that has its own set of advantages and disadvantages (just like
      all other numerical methods and techniques).

      Regarding doubt about the performance results presented in Figure 6 
      (previous Figure 8).  The results are accurate.  While it is true that
      the computation time as a function of the number of grid points is 
      higher for OTS-NIDC schemes, the computation time as a function of the
      numerical erorr is lower for OTS-NIDC schemes.
    
      I have made the the MATLAB source code (and pre-generated data) for all of
      the plots in the manuscripts available on the web at

\begin{verbatim}
http://ktchu.serendipityresearch.org/download/research/applied_math/OTS-NIDC/
\end{verbatim}

      I will leave the source code in this location (or provide a redirect if it
      moved) until the review of the manuscript is complete.  I encourage the
      reviewers to look through and point out any errors that I may have
      accidentally overlooked.  I have made an effort to make the code readable
      but apologize in advance if the implementation is not completely
      transparent.
    
      For what it's worth, I am of the opinion of the opinion that source code
      for numerical mathematics and computational science results be part of the
      manuscript review process (and even publication).  Unfortunately, this is
      not currently a standard part of the review process.

\item I have added the requested result to the manuscript. The results show
      that for a desired accuracy level, OTS-NIDC does indeed outperform a
      conventional forward Euler schemea even with the additional cost of the
      defect correction terms.

\item Noted and reworded to acknowledge that there has been other work in the
      area of optimal numerical parameter selection in recent years.  If there
      are specific examples that Reviewer \#3 feels should be referenced, I
      would be happy to include them.

\item Fourth-order accuracy of the result in Figure 8 (previously Figure 11)
      results from the use of fourth-order accurate ghost cell values when
      imposing boundary conditions.  The details of how the ghost cell values
      are calculated is discussed in Section 5.3.1 (previously Section 4.3.1).

\end{enumerate}

\subsection*{Reviewer \#4}

\subsubsection*{Rebuttals to specific points}

\begin{enumerate}

\item I have reworked the entire manuscript to emphasize that both optimal time
      step selection \emph{and} non-iterative defect correction are required to 
      boost the order of accuracy.  

      I have added comments in the abstract and the third paragraph of the
      introduction to clarify the class of PDEs that OTS-NIDC is applicable to.
    
      One point that I would like to emphasize is that, unlike traditional
      defect correction and modified equation methods, the defect correction
      method is non-iterative and easy to incorporate into the software
      implementation.  As far as I can tell, the non-iterative approach is
      somewhat novel or less common.

\item I have changed the comments applicability to GPU computing to be a 
      potential area of future research rather than presenting it as a
      conclusion drawn from existing work.
    
\item I have streamlined the entire manuscript to remove unnecessary detail,
      remove overly pedantic discussion and consolidate figures to reduce the
      length of the manuscript.  The new manuscript is about 4 pages shorter
      than the original manuscript but should still contain the same message
      as before.

\end{enumerate}

\end{document}
