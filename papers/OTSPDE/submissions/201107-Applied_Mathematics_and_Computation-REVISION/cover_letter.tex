\documentclass[12pt]{letter}

% page settings
\textwidth 6.5in
\textheight 9in
\oddsidemargin -0.05in
\evensidemargin -0.05in
\topmargin -0.85in

\begin{document}

\address{}
\signature{Kevin T. Chu}

\begin{letter}
{
% NORMALLY ADDRESS GOES HERE
}

\opening{Dear Applied Mathematics and Computations Editor/Reviewer,}

I have revised my manuscript titled \emph{Boosting the Accuracy of Finite
Difference Schemes via Optimal Time Step Selection and Non-Iterative Defect
Correction} in response to the suggestions and comments of the anonymous
reviewers.  The manuscript discusses a simple, systematic technique for
boosting the order of accuracy of finite difference schemes for time dependent,
semilinear, scalar PDEs with constant coefficient leading-order spatial
derivative term by choosing an optimal, \emph{fixed} time step and, when 
necessary, adding defect correction terms in a \emph{non-iterative} manner. 
As discussed in the paper, a careful choice of time step and correction terms
can reduce the order of the numerical error of simple finite difference
schemes and make it possible to achieve high-order accuracy using only
formally low-order finite difference stencils and time integration schemes.  

This paper describes the method for calculating the optimal time step and
defect correction terms and explains the observed boost in accuracy using 
straightforward numerical analysis arguments.  The technique of optimal time 
step selection with non-iterative defect correction (OTS-NIDC) is applicable
to several types of finite difference schemes for linear and semilinear PDEs
in any number of space dimensions and both regular and irregular domains.
Examples of each of these types of problems are discussed and analyzed in the
paper.

Thank you for your review of the paper.  I look forward to hearing your 
comments and suggestions for improvement of the manuscript.

\closing{Sincerely,}

\end{letter}

\end{document}
