\documentclass[12pt]{letter}

% page settings
\textwidth 6.5in
\textheight 9in
\oddsidemargin -0.05in
\evensidemargin -0.05in
\topmargin -0.85in

\begin{document}

\address{}
\signature{Kevin T. Chu}

\begin{letter}
{
% NORMALLY ADDRESS GOES HERE
}

\opening{Dear Applied Mathematics and Computations Editor/Reviewer,}

My article titled \emph{Using Optimal Time Step Selection to Boost the
Accuracy of Finite Difference Schemes for Time Dependent PDEs} discusses a 
simple, systematic technique for boosting the order of accuracy of finite
difference schemes for time dependent PDEs by choosing an optimal, 
\emph{fixed} time step (and possibly adding defect correction terms in a 
\emph{non-iterative} manner).  As discussed in the paper, a careful choice of
time step can reduce the order of the numerical error of simple finite 
difference schemes and make it possible to achieve high-order accuracy using 
only formally low-order finite difference stencils and time integration 
schemes.  

This paper describes the method for calculating the optimal time step and
defect correcction terms and explains the observed boost in accuracy using 
straightforward numerical analysis arguments.  The technique of optimal time 
step selection is applicable to several types of finite difference schemes
for linear and semilinear PDEs in any number of space dimensions and both 
regular and irregular domains.  Examples of each of these types of problems 
are discussed and analyzed in the paper.

In addition to being an interesting method, I believe that this method has
value in light of recent advances in GPGPU computing.  Because modern GPU
hardware is more compatible with explicit time stepping schemes, the ability
of optimal time step selection to dramaticaly boost the accuracy of simple
explicit schemes makes it a powerful tool for achieving high-performance
on GPUs.

Thank you for your review of the paper.  I look forward to hearing your 
comments and suggestions for improvement of the manuscript.

\closing{Sincerely,}

\end{letter}

\end{document}
