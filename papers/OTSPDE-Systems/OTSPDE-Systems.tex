%%%%%%%%%%%%%%%%%%%%%%%%%%%%%%%%%%%%%%%%%%%%%%%%%%%%%%%%%%%%
% Manuscript started: 2008/09/0?
% First completed draft: 2008/??/??, Vitamin D, Inc.
% Submission draft: 2008/??/??, Vitamin D, Inc.
%%%%%%%%%%%%%%%%%%%%%%%%%%%%%%%%%%%%%%%%%%%%%%%%%%%%%%%%%%%%

\documentclass[oneeqnum,onefignum,onetabnum,onethmnum]{siamltex}
\usepackage{graphics}
\usepackage{multirow}

% Macros for entire document

% Utility commands
\newcommand{\bc}{\begin{center}}
\newcommand{\ec}{\end{center}}
\newcommand{\beq}{\begin{equation}}
\newcommand{\eeq}{\end{equation}}
\newcommand{\bea}{\begin{eqnarray}}
\newcommand{\eea}{\end{eqnarray}}
\newcommand{\ba}{\begin{array}}
\newcommand{\ea}{\end{array}}

% Fancy equation environments
\newcommand{\Eq}[2][Eq.~]{{#1}(\ref{eq:#2})}                                    
\newcommand{\Eqs}[1]{Eqs.~(\ref{eq:#1})}                                        
\newcommand{\Fig}[2][Fig.~]{{#1}\ref{fig:#2}}                                   
\newcommand{\Figs}[1]{Figs.~\ref{fig:#1}}                                       
\newcommand{\Sec}[2][Sec.~]{{#1}\ref{sec:#2}}                                   
\newcommand{\Secs}[1]{Secs.~\ref{sec:#1}}      

% Common expressions
\def\eg{\emph{e.g., }}
\def\ie{\emph{i.e., }}
\def\etal{\emph{et al.}}
\def\etc{\emph{etc.}}

% Mathematical notation
\def\div{\ensuremath{\nabla \cdot}}
\def\divs{\ensuremath{\nabla_{s} \cdot}}
\def\grad{\ensuremath{\nabla}}
\def\grads{\ensuremath{\nabla_{s}}}
\def\lapl{\ensuremath{\nabla^2}}
\def\Real{\ensuremath{\mathrm{Re}}}
\newcommand{\tensor}[1]{\ensuremath{{\bf{#1}}}}

\newcommand{\ddx}[2]{\frac{\partial #1}{\partial #2}}
\newcommand{\DDx}[2]{\frac{D #1}{D #2}}
\newcommand{\sgn}[1]{\ensuremath{\mathrm{sgn}(#1)}}
\newcommand{\abs}[1]{\ensuremath{\left|#1\right|}}


% MATH MACROS
\def\sech{\mathrm{sech}}
\def\erfc{\mathrm{erfc}}

% PDE MACROS
\def\pt{\partial t}
\def\px{\partial x}
\def\py{\partial y}
\def\tu{\tilde{u}}

% NUMERICS MACROS
\def\dt{\Delta t}
\def\dx{\Delta x}
\def\dy{\Delta y}
\def\dy{\Delta y}
\def\dto{\dt_{opt}}


\title{High-Order Accurate Finite Difference Schemes \\
       for Systems of PDEs via Optimal Time Step Selection}

\author{
Kevin T. Chu\footnotemark[2] \footnotemark[3]
}

\begin{document}
\bibliographystyle{unsrt}
\maketitle

\renewcommand{\thefootnote}{\fnsymbol{footnote}}
\footnotetext[2]{Vitamin D, Inc., Menlo Park, CA 94025} 

\renewcommand{\thefootnote}{\fnsymbol{footnote}}
\footnotetext[3]{Institute of High Performance Computing, A*STAR, Singapore, Singapore} 

\renewcommand{\thefootnote}{\arabic{footnote}}


\begin{abstract}
place holder...
\end{abstract}


\begin{keywords}
optimal time step, finite difference schemes, high-order accurate numerical 
methods, time dependent PDEs, systems of PDEs
\end{keywords}

\begin{AMS}
65-02, 65M06, 65M12, 65M20, ??
\end{AMS}

\pagestyle{myheadings}
\thispagestyle{plain}
\markboth{KEVIN T. CHU}
         {HIGH-ORDER FD SCHEMES VIA OPTIMAL TIME STEPS}


\section*{Introduction}
High-order numerical methods for partial differential equations (PDEs) will 
always be valuable for increasing the computational efficiency of numerical 
simulations.  Thus, it is not at all surprising that a great deal of effort in 
numerical PDEs continues to be focused on the development of high-order 
numerical schemes~\cite{bruger_2005, gibou_2005, ito_2005, shukla_2005, 
shukla_2007}.  
Typically, high-order accuracy is achieved by constructing
schemes that have high formal orders of accuracy.  However, high-order 
accurate numerical solutions can also be obtained by using formally low-order 
schemes in clever ways.  When possible, the latter approach can be a powerful 
way to boost the accuracy of a numerical method \emph{without} introducing too 
much additional algorithmic (and programming) complexity.

\section{\label{sec:summary} Summary} 


\section*{Acknowledgments}
The author gratefully acknowledges the support of Vitamin D, Inc.
and the Institute for High-Performance Computing (IHPC) in Singapore. 
The author would like to thank K.-H. Chiam, ?? 
for helpful suggestions on the manuscript.  
The author particularly thanks K.-H. Chiam for pointing him to the 
mathematical biology literature for interesting examples of reaction-diffusion 
systems.

\bibliography{OTSPDE-Systems}

\end{document}
